\section{Research Plan }\label{research-plan}


\subsection{Objective:}\label{objective}
  
My long-term goal is to develop and validate a quantitative measure of healthy aging that depends solely on structured data universally available in electronic medical record systems or that can be linked to them (e.g. census social security data). Such a measure would be important for research because it could provide rapid and granular feedback about the effectiveness of anti-frailty interventions instead of depending on measuring time elapsed until an event such as mortality or loss of independence. 

My short-term goal for this study is to establish that, proper adjustment of coefficients will improve the predictive accuracy of mFI as well as its agreement with RAI-A. I will then determine whether mFI, as adapted to NSQIP \cite{Rockwood_2005}, is also an accurate predictor of two long-term outcomes that I have already linked to patient data: income from census data, which I have linked on patient addresses and mortality from the social security death master file which I have linked on social security numbers (in addition to some deaths already being reported directly in the EMR). Finally, I will determine whether an mFI analogue calculated on EMR data correlates with a NSQIP-based mFI on the same set of patients, and whether it exhibits convergent validity for predicting long-term outcomes on those patients. 

The ow

\subsection{Rationale:}\label{rationale}

\subsection{Specific Aims:}\label{specific-aims}
