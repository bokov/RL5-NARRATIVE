\section{Research Plan }\label{research-plan}


\subsection{Objective:}\label{objective}
  
My long-term goal is to develop and validate a quantitative measure of healthy aging that depends solely on structured data universally available in electronic medical record systems or that can be linked to them (e.g. census and social security data). Such a measure would be important for research because it could provide rapid and granular feedback about the effectiveness of anti-frailty interventions instead of depending on measuring time elapsed until an event such as mortality or loss of independence. 

My short-term goal for the coming two years is to establish that a statistically valid adjustment of coefficients will improve the predictive accuracy of mFI as well as its agreement with RAI-A \cite{Isharwal_2016,Melin_2015}. I will then determine whether mFI \cite{Rockwood_2005}, as adapted to NSQIP \cite{Tsiouris_2013}, is also an accurate predictor of two long-term outcomes: nursing home placement based on last valid patient address and mortality from the social security death master file (in addition to some deaths already being reported directly in the EMR). I have already, during my collaboration with Drs. Shireman and Espinosa, linked both variables to over 900,000 de-identified unique patient records extracted from UT Medicine and University Health System. Finally, I will determine whether an mFI analogue calculated on EMR data correlates with a NSQIP-based mFI on the same set of patients, and whether it exhibits convergent validity for predicting long-term outcomes on those patients. 

The work I propose will create the means by which mFI, a frailty measure used by geriatricians as well as surgeons, could be calculated for any patient, not only those selected for inclusion into the NSQIP registry. Furthermore, it could be calculated at any time, not just pursuant to surgery, and without relying on an in-person assessment, performance test, or questionnaire.

\subsection{Rationale:}\label{rationale}
In 2013, an international expert panel organized by the American Medical Directors Association and including among its members Linda Fried and Ken Rockwood, originators of the two main schools of thought in frailty metrics field, published a consensus report \cite{Morley_2013} stating that frailty is a clinically important syndrome that is tractable to interventions and recommending that all patients older than 70 be screened for frailty. Individuals identified as frail are especially likely benefit from individualized exercise regimens, anti-inflammatory drugs, and closer coordination between specialists and primary care physicians \citep{16392724}. [TODO: mine 1st national frailty workshop whitepaper for more benefits of frailty screening]

Physical assessment is time-consuming and some measures such as timed up-and-go or walking speed \cite{10811152} could place stress on already vulnerable patients. Furthermore, if such data is recorded as free-text physician notes, it is difficult to analyze later. If it is recorded in some external tool such as REDCap \cite{Harris_2009} it becomes difficult to unobtrusively integrate this information back into the clinical workflow. Problematic though "alert-fatigue" is \cite{Kesselheim_2011}, one can only imagine that having to repeatedly copy-paste between an EMR screen and some web-based instrument can only be worse. 

The obvious-seeming solution (to anybody who has not tried to actually interact with a real-life EMR system or its real-life users) would be to take the terabytes of patient data collected in the normal course of healthcare delivery, and use it to learn something about the patients. Yet a recent systematic review of frailty assessment tools \cite{McDonald_2016} could only find 42 distinct such tools, of which only four met their criteria of feasibility, objectivity, and utility. Of these four, one was the modified frailty index (mFI) \cite{Tsiouris_2013,22491599,23711971,Hodari_2013,Farhat_2012,23949353} a NSQIP-based measure that is widely used in surgery outcomes research and will be discussed below. A second was simply determining whether the patient had a history of falling. Only the remaining two approaches actually attempted to base

Ideally an algorithm would calculate the initial frailty score for every patient based on their electronic medical record. Patients with scores outside the reference range would be given physical assessments or interviewed to confirm the automatically calculated score.  

\subsection{Specific Aims:}\label{specific-aims}
