\section{Research Plan }\label{research-plan}


\subsection{Objective:}\label{objective}
  
My long-term goal is to develop and validate a quantitative measure of healthy aging that depends solely on structured data universally available in electronic medical record systems or that can be linked to them (e.g. census and social security data). Such a measure would be important for research because it could provide rapid and granular feedback about the effectiveness of anti-frailty interventions instead of depending on measuring time elapsed until an event such as mortality or loss of independence. 

My short-term goal for the coming two years is to establish that a statistically valid adjustment of coefficients will improve the predictive accuracy of mFI as well as its agreement with RAI-A \cite{Isharwal_2016,Melin_2015}. I will then determine whether mFI \cite{Rockwood_2005}, as adapted to NSQIP \cite{Tsiouris_2013}, is also an accurate predictor of two long-term outcomes: nursing home placement based on last valid patient address and mortality from the social security death master file (in addition to some deaths already being reported directly in the EMR). I have already, during my collaboration with Drs. Shireman and Espinosa, linked both variables to over 900,000 de-identified unique patient records extracted from UT Medicine and University Health System. Finally, I will determine whether an mFI analogue calculated on EMR data correlates with a NSQIP-based mFI on the same set of patients, and whether it exhibits convergent validity for predicting long-term outcomes on those patients. 

\subsection{Rationale:}\label{rationale}
In 2013, an international expert panel organized by the American Medical Directors Association and including among its members Linda Fried and Ken Rockwood, originators of the two main schools of thought in frailty metrics field, published a consensus report \cite{Morley_2013} stating that frailty is a clinically important syndrome that is tractable to interventions and recommending that all patients older than 70 be screened for frailty. Individuals identified as frail are especially likely benefit from individualized exercise regimens, anti-inflammatory drugs, and closer coordination between specialists and primary care physicians \citep{16392724}. [\textbf{TODO: mine 1st national frailty workshop whitepaper for more benefits of frailty screening, especially at the intersection of surgery and geriatrics}]

Physical assessment is time-consuming and some measures such as timed up-and-go or walking speed \cite{10811152} could place stress on already vulnerable patients. Furthermore, if such data is recorded as free-text physician notes, it is difficult to analyze later. If it is recorded in some external tool such as REDCap \cite{Harris_2009} it becomes difficult to unobtrusively integrate this information back into the clinical workflow. Problematic though "alert-fatigue" is \cite{Kesselheim_2011}, one can only imagine that having to repeatedly copy-paste between an EMR screen and some web-based instrument can only be worse. 

The obvious-seeming solution (to anybody who has not tried to actually interact with a real-life EMR system or its real-life users) would be to take the terabytes of patient data collected in the normal course of healthcare delivery, and use it to learn something about a patient \textit{before} even meeting them in person. The deficit-accumulation family of frailty metrics \cite{Rockwood_2005} in particular would seem like a ripe target for developing an EMR alert, since it is very easy to calculate and robust against missing data. Yet a recent systematic review of frailty assessment tools \cite{McDonald_2016} could only find 42 distinct such tools, of which only four met their criteria of feasibility, objectivity, and utility. Of these four, one was the modified frailty index (mFI) \cite{Tsiouris_2013,22491599,23711971,Hodari_2013,Farhat_2012,23949353} a NSQIP-based deficit-accumulation measure widely used in surgery outcomes research and will be discussed below. The second was simply determining whether the patient had a history of falling \cite{Jones_2013}. Only the remaining two approaches \cite{Patel_2013,Amrock_2014} used actual EMR systems to calculate frailty scores (both used ). Both were limited to specific types of surgery and only the second used a large demographically diverse sample. Interestingly, that study also demonstrated that there was little practical advantage to limiting one's self only to patients in the NSQIP registry.

Why have there not been more attempts to utilize the actual EMR systems? One reason is that EMR systems are complicated, opaque, and to use them effectively in research close collaboration between three highly specialized domains is needed: computer programming, statistics, and medicine. In contrast, registries like ACS NSQIP offer the advantage of high quality expert-curated data with clearly defined guidelines and good inter-rater reliability \cite{Shiloach_2010}. But beyond the proof-of-principle stage, registry data is of limited use in clinical decision support: it includes only a subset of patients from each participating site, it includes only a pre-determined set of variables, and most importantly, it can only be used retrospectively after a registrar has performed the necessary chart abstraction. In other words, predictors bound to registries cannot be used to create real-time alerts recommending, for example, less invasive treatment options or a geriatric consult prior to surgery. Nevertheless, ACS NSQIP can be a good proving ground for algorithms that can then be migrated to the electronic medical record (EMR). First, a copy of a health-system's local NSQIP registry is obtained \textit{with} patient identifiers still included (as I have done in support of Dr. Shireman's Pepper Center Pilot project on frailty as a predictor of post-operative complications in older patients). Then, a frailty measure previous validated on NSQIP data is re-created; in this case the frailty measure was the Administrative Risk Analysis Index (RAI-A) \cite{Hall_2017,Johnson_2014,Melin_2015,Gupta_2011}. Next, under the supervision of an experienced clinician and reference to the NSQIP Participant Use Data File (PUF), the equivalent diagnosis codes (ICD10) and laboratory values need to be matched in the EMR system and adjusted until their predictions converge the registry-based ones, for the same respective patients.

Though Dr. Shireman's and Dr. Espinoza's Pilot Project and the R21 that resulted from it rely on the RAI-A, for my own project I intend to apply what I learned to the Rockwood Index because it is more robust than RAI-A, is in use by both geriatricians and surgeons [todo: find refs for how geriatricians actually use it to collaborate with surgeons], and has a statistically sound grounding in bio-gerontology -- briefly, a patient's Rockwood Index can be used to construct a hazard model  \cite{Mitnitski_2001} similar to the Gompertz function \cite{Mueller_1995,Pletcher_2000,Bokov_2017}.

A widely used variant of the Rockwood Index used by surgeons and validated against NSQIP is called the modified Frailty Index (mFI) \cite{Uppal_2015,Tsiouris_2013,Bellamy_2017,22491599,23711971,23949353}. Not all sites participating in NSQIP support all the variables used by mFI, including ours. However, Rockwood has also published a \textit{methodology} for developing a frailty index that has a high degree of agreement with other indexes developed using the same methodology. This, for the purposes of this research project, is what I mean by the Rockwood Index (for preliminary version of this index have already started testing, see Research Plan section below).


\subsection{Specific Aims}\label{specific-aims}

\subsubsection{Specific Aim 1}\label{aim1} To finish development and internal validation of the preliminary Rockwood Index I developed using NSQIP data. This will require the following steps:
\begin{itemize}
\item Under the guidance of my geriatric and surgical mentors, internal EMR value-flags, and the relevant literature, I will identify age-adjusted reference ranges for the NSQIP laboratory values and use them to dichotomize those values into normal/abnormal, so that they can also be included in the Rockwood Index.
\item I will determine whether there is convergent validity with RAI-A and predictive validity for postoperative complications and readmissions.
\item I will determine whether the predictions and parameter distributions are stable under resampling. I will then further validate the predictions of my model against the hold-out validation subset of the data.
\end{itemize}

\subsubsection{Specific Aim 2}\label{ To link our NSQIP dataset to long-term outcomes that are relevant to healthy aging and determine whether the Rockwood index needs to be adjusted for socioeconomic covariates. In particular:
\begin{itemize}
\item To build on the my previous EMR geocoding work by matching patient address histories to the addresses of nursing homes and hospices, so that the transition from private addresses could serve as  proxy variable for loss of independence.
\item To validate and refine the address-matching rules using as a standard the 554 NSQIP patients with complete records and discharge destinations other than their homes.
\item To determine which, if any, adjustments for patient ethnicity or income will further improve predictive accuracy.
\end{itemize}

\emph{Specific Aim 3:} To migrate the Rockwood Index to the EMR system and demonstrate that it remains consistent with the NSQIP version. This will entail the following steps:
\begin{itemize}
\item To work with my geriatric and surgical mentor/s to identify groups EMR data elements that reliably correspond to NSQIP variables.
\item To determine whether there strong agreement between the NSQIP and EMR versions of the Rockwood Index, and whether the EMR version has improved prediction of long-term survival and independence compared to the NSQIP version.
\item To determine whether the Rockwood Index is effective in predicting other well-established sequelae of frailty (falls and lower limb fractures, pneumonia, outpatient readmissions, and death) in a the larger population of the EMR and no longer limited to surgery cases.
\end{itemize}
