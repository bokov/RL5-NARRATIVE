
I also followed the process for developing a frailty measure recommended by Rockwood \cite{Rockwood_2005}. This frailty metric is even more strongly correlated with both outcomes and even in this small sample has a similar probability distribution to the original frailty measure derived by Rockwood \cite{Mitnitski_2001} in a completely different dataset with different component variables, but chosen using the same methodology.



\begin{table}\centering 
  \caption{ # Postoperative Complications vs. Rockwood }\label{wrap-tab:4} 
\begin{tabular}{@{\extracolsep{5pt}}lD{.}{.}{-3} } 
\\[-1.8ex]\hline 
\hline \\[-1.8ex] 
 & \multicolumn{1}{c}{\textit{Dependent variable:}} \\ 
\cline{2-2} 
\\[-1.8ex] & \multicolumn{1}{c}{Post-Op Complications} \\ 
\hline \\[-1.8ex] 
 Rockwood & 10.878$^{***}$ (2.319) \\ 
  & \\ 
 Income & -0.00001$^{***}$ (0.00000) \\ 
  & \\ 
 Age & 0.021$^{***}$ (0.006) \\ 
  & \\ 
 Rockwood x Age & -0.120$^{***}$ (0.040) \\ 
  & \\ 
 Rockwood x Income & 0.00004 (0.00003) \\ 
  & \\ 
 Constant & -1.900$^{***}$ (0.365) \\ 
  & \\ 
\hline \\[-1.8ex] 
Observations & \multicolumn{1}{c}{757} \\ 
Log Likelihood & \multicolumn{1}{c}{-986.191} \\ 
Akaike Inf. Crit. & \multicolumn{1}{c}{1,984.382} \\ 
\hline 
\hline \\[-1.8ex] 
  & \multicolumn{1}{r}{$^{*}$p$<$0.1; $^{**}$p$<$0.05; $^{***}$p$<$0.01} \\ 
\end{tabular} 
\end{table} 

[TABLE 5 Death or Readmission vs RAI-A] [Figure 2]

The following non-acute pre-operative NSQIP variables were used for this Rockwood Index: [...,...,...]

\paragraph{Study Design}\label{study-design}


\paragraph{Study Population}\label{study-population}


\paragraph{Study Procedures}\label{study-procedures}


\paragraph{Data Analysis}\label{data-analysis}

\paragraph{Power and Sample Size}\label{power-and-sample-size}

\paragraph{Expected Outcomes}\label{expected-outcomes}

\paragraph{Potential Problems and Alternative
      Approaches}\label{potential-problems-and-alternative-approaches}

\paragraph{Future Directions}\label{future-directions} 
    
\section{(Temporary Storage, Not Part of Final Document)}\label{ammo}
There are two main types of frailty scores used in geriatric research and clinical practice: the Fried frailty phenotype \cite{Fried_2001} and the Rockwood Frailty Index \cite{Mitnitski_2001}. The frailty phenotype is anchored on weakness, sarcopenia, diminished mobility as recorded via patient questionnaire or an physical assessment performed by a physician. The frailty index is based on the deficit accumulation theory and is calculated as a simple unweighted sum of selected diagnoses, conditions, and dichotomized lab values, corrected for missing observations. The actual choice of variables is less important than them spanning multiple clinical domains \cite{Searle_2008}. 

\subsection{Link to Aging}\label{link-to-aging}

\subsection{Clinical Impact}\label{clinical-impact}


