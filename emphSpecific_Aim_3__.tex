\emph{Specific Aim 3:} To map each of the 11 NSQIP variables used by mFI to ICD10 codes and lab values in the electronic medical record system.


\subsection{Research Strategy}\label{research-strategy}


\subsubsection{Significance}\label{significance}

\subsubsection{Innovation}\label{innovation}

\paragraph{Preliminary Studies and Feasibility}\label{preliminary-studies-and-feasibility} This is a test.


\paragraph{Study Design}\label{study-design}


\paragraph{Study Population}\label{study-population}


\paragraph{Study Procedures}\label{study-procedures}


\paragraph{Data Analysis}\label{data-analysis}

\paragraph{Power and Sample Size}\label{power-and-sample-size}

\paragraph{Expected Outcomes}\label{expected-outcomes}

\paragraph{Potential Problems and Alternative
      Approaches}\label{potential-problems-and-alternative-approaches}

\paragraph{Future Directions}\label{future-directions} 
    
\subsection{Cited Literature}\label{cited-literature}


\section{Ammo (Not Part of Final Document)}\label{ammo}

\subsection{Link to Aging}\label{link-to-aging}

\subsection{Clinical Impact}\label{clinical-impact}
There are two main types of frailty scores used in geriatric research and clinical practice: the Fried frailty phenotype \cite{Fried_2001} and the Rockwood Frailty Index \cite{Mitnitski_2001}. The frailty phenotype is anchored on weakness, sarcopenia, diminished mobility as recorded via patient questionnaire or an physical assessment performed by a physician. The frailty index is based on the deficit accumulation theory and is calculated as a simple unweighted sum of selected diagnoses, conditions, and dichotomized lab values, corrected for missing observations. The actual choice of variables is less important than them spanning multiple clinical domains \cite{Searle_2008}. In fact Rockwood indices constructed by \textit{randomly} selecting from a set of facially valid variables exhibit a high degree of agreement with each other \cite{Mitnitski_2001}.

The modified frailty index (mFI) is a variant of the Rockwood Index which uses 11 variables from the American College of Surgeons National Quality Improvement Program (ACS NSQIP) registry, and has been effectively used to predict complications for many different types of surgery \cite{Uppal_2015,Tsiouris_2013,Bellamy_2017,22491599,23711971,23949353}. Registries such as ACS NSQIP offer the advantage of high quality expert-curated data with clearly defined guidelines and good inter-rater reliability \cite{Shiloach_2010}. Yet curated registry data has the following limitations in their usefulness to research and clinical decision support: it includes only a limited subset of patients from each participating site, it includes only a pre-determined set of variables, and most importantly, it can only be interpreted retrospectively after a registrar has performed the necessary chart abstraction. Therefore, predictors bound to registries cannot be used to create real-time alerts recommending, for example, less invasive treatment options or a geriatric consult prior to surgery.

However, ACS NSQIP can be a good proving ground for algorithms that can then be migrated to the electronic medical record (EMR). First, a copy of a health-system's local NSQIP registry is obtained \textit{with} patient identifiers still included (as I have done in support of Dr. Shireman's Pepper Center Pilot project on frailty as a predictor of post-operative complications in older patients). Then, a frailty measure previous validated on NSQIP data is re-created; in this case the frailty measure was the Administrative Risk Analysis Index (RAI-A) \cite{Hall_2017,Johnson_2014,Melin_2015,Gupta_2011}. Next, under the supervision of an experienced clinician and reference to the NSQIP Participant Use Data File (PUF), the equivalent diagnosis codes (ICD10) and laboratory values need to be matched in the EMR system and adjusted until their predictions converge the registry-based ones, for the same respective patients.

Though Dr. Shireman's and Dr. Espinoza's Pilot Project and the R21 that resulted from it are centered around the RAI-A, for my own project I intend to apply what I learned to the Rockwood Index because it is more robust than RAI-A, is in use by both geriatricians and surgeons [todo: find refs], and has a statistically sound grounding in bio-gerontology -- briefly, a patient's Rockwood Index can be used to construct a hazard model  \cite{Mitnitski_2001} similar to the Gompertz function \cite{Mueller_1995,Pletcher_2000,Bokov_2017}.

\emph{Specific Aim 1:} To finish development and internal validation of the preliminary Rockwood Index I developed using NSQIP data.
\emph{Specific Aim 2:} To link our NSQIP dataset to long-term outcomes that are relevant to healthy aging as well as to socioeconomic covariates. In particular:
\begin{itemize}
\item To build on the EMR geocoding work I have done already to match patient address histories to the addresses of nursing homes and hospices, so that the transition from private addresses could serve as  proxy variable for loss of independence.
\item To validate and refine the address-matching rules using as a standard the 554 NSQIP patients with complete records and discharge destinations other than their homes.
\item 
\end{itemize}
