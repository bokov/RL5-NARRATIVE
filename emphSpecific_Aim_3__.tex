\subsection{Research Strategy}\label{research-strategy}


\subsubsection{Significance}\label{significance}

\subsubsection{Innovation}\label{innovation}

\paragraph{Preliminary Studies and Feasibility}\label{preliminary-studies-and-feasibility} 
Since starting my first faculty position at this University in 2014, I have played a key role in creating a de-identified i2b2 data warehouse that now contains de-identified electronic health records for 383,752 patients with outpatient visits recorded in UT Health's Epic EMR. As of this year the final regulatory hurdles have been overcome to merging these records with almost 700,000 new patients who have inpatient, emergency, or outpatient visits recorded in the Sunrise EMR of University Health System (UHS), our main teaching hospital partner. A demographic breakdown of our patient population is shown in Table 1 [\textbf{TODO: generate table }] alongside the corresponding counts for patients that are in the UHS NSQIP registry.

This year I linked the addresses of over 800,000 (75\%) of the patients from the combined UHS/UT Health population to census block groups and then used those census block group to assign an estimated median household income to each of those patients. The i2b2 data warehouse that I helped create also contains long-term mortality data on 5971 patients as recorded in the EMR system and 8406 additional patients from social security data that would \textit{not} have been available from the EMR system alone. This fall, as part of my ongoing collaboration with Dr. Espinoza and Dr. Shireman I will load an estimated 8,000-30,000 additional unique death dates from SSDMF updates that I have already obtained. When I previously applied for this RL5 award I believe I may have lost points on the grounds of feasibility because reviewers unfamiliar with my i2b2 work might not have understood the scope of what I have \textit{already} done so I want to make sure that I address this concern explicitly and decisively.

Also as part of our collaboration, Dr. Shireman and I obtained an identified copy of the UHS ACS NSQIP registry going back to April of 2013. It contains records on 6408 surgeries on 5914 distinct patients. Of these, 3990 are Hispanic and 1396 are non-Hispanic Whites and together comprise the eligibility set for Dr. Espinoza's and Dr. Shireman's Pilot Project (N=5386). Of these, 4459 have incomes inferred from census data due to the geocoding efforts described above. From this dataset, in turn, I have randomly sampled a developmental dataset of 905 patients. Excepting high-level cohort characterization such as shown in Table 1, I will make all statistical decisions based only on this developmental sample, deliberately blinding myself to the validation hold-out sample. Only when model specification and variable selection are complete will I re-run my analysis on the remaining data for purposes of formal hypothesis testing and predictive validation. The preliminary data I am showing here is from the developmental sample.

I have calculated a partial RAI-A score (partial because )


\paragraph{Study Design}\label{study-design}


\paragraph{Study Population}\label{study-population}


\paragraph{Study Procedures}\label{study-procedures}


\paragraph{Data Analysis}\label{data-analysis}

\paragraph{Power and Sample Size}\label{power-and-sample-size}

\paragraph{Expected Outcomes}\label{expected-outcomes}

\paragraph{Potential Problems and Alternative
      Approaches}\label{potential-problems-and-alternative-approaches}

\paragraph{Future Directions}\label{future-directions} 
    
\section{(Temporary Storage, Not Part of Final Document)}\label{ammo}
There are two main types of frailty scores used in geriatric research and clinical practice: the Fried frailty phenotype \cite{Fried_2001} and the Rockwood Frailty Index \cite{Mitnitski_2001}. The frailty phenotype is anchored on weakness, sarcopenia, diminished mobility as recorded via patient questionnaire or an physical assessment performed by a physician. The frailty index is based on the deficit accumulation theory and is calculated as a simple unweighted sum of selected diagnoses, conditions, and dichotomized lab values, corrected for missing observations. The actual choice of variables is less important than them spanning multiple clinical domains \cite{Searle_2008}. 

\subsection{Link to Aging}\label{link-to-aging}

\subsection{Clinical Impact}\label{clinical-impact}


