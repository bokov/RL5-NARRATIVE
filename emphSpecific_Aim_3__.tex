\emph{Specific Aim 3:} To map each of the 11 NSQIP variables used by mFI to ICD10 codes and lab values in the electronic medical record system.


\subsection{Research Strategy}\label{research-strategy}


\subsubsection{Significance}\label{significance}

\subsubsection{Innovation}\label{innovation}

\paragraph{Preliminary Studies and Feasibility}\label{preliminary-studies-and-feasibility} This is a test.


\paragraph{Study Design}\label{study-design}


\paragraph{Study Population}\label{study-population}


\paragraph{Study Procedures}\label{study-procedures}


\paragraph{Data Analysis}\label{data-analysis}

\paragraph{Power and Sample Size}\label{power-and-sample-size}

\paragraph{Expected Outcomes}\label{expected-outcomes}

\paragraph{Potential Problems and Alternative
      Approaches}\label{potential-problems-and-alternative-approaches}

\paragraph{Future Directions}\label{future-directions} 
    
\subsection{Cited Literature}\label{cited-literature}


\section{Ammo (Not Part of Final Document)}\label{ammo}

\subsection{Link to Aging}\label{link-to-aging}

\subsection{Clinical Impact}\label{clinical-impact}
There are two main types of frailty scores used in geriatric research and clinical practice: the Fried frailty phenotype \cite{Fried_2001} and the Rockwood Frailty Index \cite{Mitnitski_2001}. The frailty phenotype is anchored on weakness, 
